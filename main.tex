\documentclass[11pt,a4paper]{article}
\usepackage[utf8]{inputenc}
\usepackage[T1]{fontenc}
\usepackage[ngerman, english]{babel}
\usepackage{csquotes}
\usepackage{hyperref}
\usepackage{graphicx}
\usepackage{enumitem}
\usepackage{biblatex}
\addbibresource{rdforms.bib}

\title{RDForms: Ein JavaScript-Framework für RDF-Formulare im Web}
\author{Dein Name -- Bachelorarbeit}
\date{\today}

\begin{document}
\maketitle

\begin{abstract}
Die Bearbeitung von RDF-Daten gilt als komplex, da RDF-Graphen für Menschen nur schwer direkt lesbar und editierbar sind. 
Mit RDForms steht ein JavaScript-Framework zur Verfügung, das RDF-Daten nutzerfreundlich über webbasierte Formulare erfassbar macht. 
Dieser Artikel beschreibt Hintergrund, Architektur und Einsatzmöglichkeiten von RDForms, vergleicht es mit verwandten Technologien und zeigt Anwendungsfelder für Wissenschaft und Praxis. 
\end{abstract}

\section{Einführung}
Das Resource Description Framework (RDF) ist eine etablierte Spezifikation zur Modellierung strukturierter, verknüpfter Daten im Kontext des Semantic Web \cite{W3C_RDF12_Concepts}. 
RDF basiert auf Tripeln (\emph{Subjekt–Prädikat–Objekt}) und ermöglicht es, Informationen maschinenlesbar und interkonnektiert darzustellen. 
Obwohl RDF eine hohe Ausdrucksmächtigkeit besitzt, stellt die Erfassung und Bearbeitung von RDF-Daten in der Praxis eine Herausforderung dar. 
Domänenexperten ohne technische RDF-Kenntnisse stoßen häufig an Grenzen, wenn sie direkt mit Serialisierungen wie Turtle oder RDF/XML arbeiten sollen.

An dieser Stelle setzt \textbf{RDForms} an: eine JavaScript-Bibliothek, die eine Brücke zwischen RDF-Datenmodellen und benutzerfreundlichen Webformularen schlägt. 
RDForms erlaubt es, Templates für Formulare zu definieren, die automatisch aus RDF-Graphen befüllt und zurückgeschrieben werden. 
Damit entsteht eine klare Trennung zwischen Datenmodell und Präsentation, ähnlich wie man es aus klassischen MVC-Architekturen kennt.

\section{Historischer Hintergrund}
Die Ursprünge von RDForms liegen in den frühen 2000er Jahren, als Projekte wie SHAME und RForms erste generische RDF-Editoren bereitstellten \cite{RDForms_Overview}. 
2008 wurde RDForms als JavaScript-Lösung vorgestellt und seit 2011 von der schwedischen Firma MetaSolutions AB weiterentwickelt. 
Heute bildet RDForms einen integralen Bestandteil der Linked-Data-Plattform \emph{EntryScape}, die insbesondere in skandinavischen Ländern in öffentlichen Verwaltungen eingesetzt wird \cite{EntryScape_Models}. 

\section{Architektur und Funktionsweise}
Die Architektur von RDForms besteht aus drei zentralen Komponenten:

\begin{itemize}[itemsep=0.5em]
  \item \textbf{Templates}: Beschreiben die Struktur des Formulars. 
  Sie geben an, welche RDF-Properties bearbeitet werden sollen, welcher Datentyp erwartet wird und welche Kardinalitäten gelten. 
  Templates werden in JSON beschrieben und können modular wiederverwendet werden.
  
  \item \textbf{Graph-Handling}: RDForms nutzt \emph{rdfjson}, um RDF-Graphen im Browser zu verwalten. 
  Ein Graph kann leer initialisiert oder aus externen Quellen (z.\,B. API, Datei) geladen werden. Änderungen durch den Nutzer werden zurück in den Graph geschrieben.
  
  \item \textbf{Renderer}: Die Templates werden durch Renderer in eine konkrete UI überführt. 
  RDForms stellt Renderer für React, Bootstrap und Material-UI bereit, wodurch sich die Formulare leicht in moderne Frontend-Frameworks integrieren lassen.
\end{itemize}

Zusätzlich existieren Validator-Komponenten, die sicherstellen, dass Kardinalitäten und Datentypen eingehalten werden. 
Dies reduziert die Gefahr fehlerhafter RDF-Daten erheblich.

\section{Beispiel}
Das folgende vereinfachte Template beschreibt eine Person mit Vorname, Nachname und Alter:

\begin{verbatim}
{
  "templates": [
    {
      "id": "foaf::givenName",
      "type": "text",
      "property": "http://xmlns.com/foaf/0.1/givenName",
      "label": {"de": "Vorname"},
      "cardinality": {"min": 1, "max": 1}
    },
    {
      "id": "foaf::familyName",
      "type": "text",
      "property": "http://xmlns.com/foaf/0.1/familyName",
      "label": {"de": "Nachname"},
      "cardinality": {"min": 1, "max": 1}
    },
    {
      "id": "schema::age",
      "type": "text",
      "property": "http://schema.org/age",
      "datatype": "xsd:integer",
      "label": {"de": "Alter"},
      "cardinality": {"max": 1}
    },
    {
      "id": "personForm",
      "type": "group",
      "items": ["foaf::givenName", "foaf::familyName", "schema::age"]
    }
  ]
}
\end{verbatim}

Dieses Template kann in einer React-Anwendung genutzt werden, um eine Person als RDF-Ressource zu bearbeiten. 
Der Renderer erzeugt automatisch Eingabefelder, die mit RDF-Tripeln verknüpft sind. 

\section{Vergleich mit Alternativen}
Einige verwandte Ansätze sind:
\begin{itemize}
  \item \textbf{XForms}: ein W3C-Standard für deklarative Formulare, jedoch ohne direkte RDF-Integration.
  \item \textbf{SHACL-Form-Engines}: nutzen SHACL-Shapes zur Validierung und Generierung von Formularen, sind aber komplexer in der Einrichtung.
  \item \textbf{DFDP (Declarative Form Description Pipeline)}: ein jüngerer Ansatz, der RDF-Editing unterstützt, allerdings weniger ausgereift ist \cite{Smessaert2024}.
\end{itemize}

Im Vergleich punktet RDForms mit einer schlanken JavaScript-Architektur, einfacher Einbettung in bestehende Webanwendungen und einem klaren Fokus auf RDF.

\section{Einsatzgebiete}
RDForms wird in verschiedenen Szenarien eingesetzt:
\begin{itemize}
  \item \textbf{Open-Data-Portale}: Formulare für die Beschreibung und Veröffentlichung von Datensätzen.
  \item \textbf{Forschungsdatenmanagement}: Eingabe von Metadaten (Autoren, Publikationen, Projekte).
  \item \textbf{Wissensgraph-Anwendungen}: Bearbeitung von Entitäten (z.\,B. Personen, Organisationen, Orte).
\end{itemize}

Durch die Entkopplung von Template und UI eignet sich RDForms besonders für wiederkehrende, standardisierte Eingabeprozesse, bei denen Konsistenz und Validität entscheidend sind.

\section{Ausblick}
Zukünftige Entwicklungen könnten eine stärkere Integration von RDForms mit \emph{Solid Pods} oder anderen verteilten Datenplattformen sein. 
Auch die automatische Generierung von Templates aus Ontologien oder SHACL-Shapes ist ein spannendes Forschungsthema. 
Dadurch könnte die manuelle Erstellung von Templates reduziert und die Wiederverwendung bestehender Wissensmodelle gefördert werden.

\section{Fazit}
RDForms stellt eine robuste und praxisnahe Lösung dar, um RDF-Daten über nutzerfreundliche Webformulare zu bearbeiten. 
Es verbindet deklarative Template-Definition, effizientes Graph-Handling und moderne UI-Integration. 
Für die Wissenschaft bietet es ein wertvolles Werkzeug, um Linked Data in interaktiven Anwendungen zugänglich zu machen. 
Im Vergleich zu Alternativen überzeugt RDForms durch Einfachheit, Flexibilität und langjährige Stabilität.

\printbibliography
\end{document}
